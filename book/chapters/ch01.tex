\setchapterstyle{kao}
\setchapterpreamble[u]{\margintoc}
\chapter{Bity i bajty}
\labch{mathematics}

\section{Wprowadzenie}

Zanim komputery osobiste weszły do powszechnego użytku,
proste obliczenia często wykonywano przy pomocy kalkulatorów
mechanicznych. Szczytowym osiągnięciem techniki z tego okresu
czasu jest Curta - przenośny kalkulator mechaniczny niewielkich rozmiarów.

% TODO: Zdjęcie curty

Kalkulatory mechaniczne często zawierały w sobie mechanizm
podobny do tego jaki znamy z analogowych wodomierzy czy hodometrów.
Jeżeli trzy cyfrowy kalkulator pokazywał by liczbę 998 to 
dodanie do niej dwukrotnie liczby jeden da nam kolejno wyniki 999 i 0.

% TODO Foto 998 -> 999 -> 000

% TODO: https://woodgears.ca/counter/
Zjawisko to określamy mianem nadmiaru (ang. overflow) i
wynika ono wprost z budowy naszego kalkulatora.

A skoro już jesteśmy przy temacie dodawania, 
to warto wspomnieć że była to jedyna operacja
którą takie kalkulatory były w stanie wykonywać.


% Plan
% 