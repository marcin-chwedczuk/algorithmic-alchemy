\setchapterstyle{kao}
\setchapterpreamble[u]{\margintoc}
\chapter{Bity i bajty}
\labch{mathematics}

\section{Wprowadzenie}

Zanim komputery osobiste weszły do powszechnego użytku,
proste obliczenia często wykonywano przy pomocy kalkulatorów
mechanicznych. Szczytowym osiągnięciem techniki z tego okresu
czasu jest Curta - kieszonkowy kalkulator mechaniczny.
Kalkulator ten pozwalał na wykonanie jedynie dwóch operacji,
wyzerowania licznika i dodania liczby (wybieranej za pomocą
zestawu suwaków umieszczonych na ścianach urządzenia) do licznika.

% TODO: Zdjęcie curty


Kalkulatory mechaniczne często zawierały w sobie mechaniczne
liczniki podobne do tych jakie znamy z analogowych wodomierzy czy hodometrów.
Liczniki takie zbudowane są z zestawu cylindrów, oraz przekładni
połączonych w taki sposób że przejście pomiędzy cyfrą dziewięć a zero
wymuszą ruch kolejnego cylindra.
% TODO: https://woodgears.ca/counter/

Ponieważ liczniki mechaniczne mają ograniczony zakres, np. licznik
składający się z trzech cylindrów może reprezentować jedynie liczby od 000 do 999,
to występuje w nich zjawisko nadmiaru (ang. overflow).
Oznacza to, że po osiągnięciu liczby 999 licznik "wraca" do liczby 000.
Okazuje się że to zjawisko ma bardzo ciekawe matematyczne konsekwencje.

% TODO Foto 998 -> 999 -> 000

Zastanówmy się kiedy obliczenia prowadzone na kalkulatorze mechanicznym,
który bazuje na mechaniźmie licznika mechanicznego są poprawne.
Oczywistym jest że obliczenie będzie poprawne jeżeli w jego trakcie
nie wystąpi nadmiar. A co się stanie gdy nadmiar wystąpi, czy wynik
jest wtedy zupełnie pozbawiony sensu? Okazuje się że wręcz przeciwnie,
bo wyświetlacz kalkulatora zawsze prezentuje trzy ostatnie cyfry obliczanej sumy.

Ubierając to w język matematyki, możemy powiedzieć że kalkulator daje
wynik który jest równy prawdziwej sumie wprowadzonych liczb modulo N.
Wyrażenie "K modulo N" oznacza że chodzi nam o wartość reszty z dzielenia
liczby K przez N. Na przykład reszta z dzielenia 12 przez 7 to 5, możemy
więc powiedzieć 12 modulo 7 to 5. Z kolei 10 modulo 5 wynosi zero, bo
10 dzieli się przez 5. Dla dodatnich liczb K i N, reszta z dzielenia K przez N
zawsze mieści się w przedziale 0..(N-1).

Możemy się teraz zastanowić czy istnieje jakiś sposób na zmuszenie
naszego kalkulatora do wykonania odejmowania.
Dla uproszczenia założymy że nasz kalkulator ma trzycyfrowy wyświetlacz.
Wiemy, po wcześniejszej analizie, że z powodu zjawiska nadmiaru
kalkulator pokaże nam 000 gdy tylko obliczana suma osiągnie wartość 1000.
Na przykład: 001 + 999 = 000. Zastanówmy się teraz co się stanie 
gdy dodamy 050 + 999 = 049 + 1000 = 049. Wygląda na to że udało nam
się wykonać odejmowanie 50 - 1 = 49. Metodę to da się uogólnić,
załóżmy że chcemy odjąć od liczby x, liczbę y (x >= y). 
Możemy to zrobić dodając na naszym kalkulatorze liczby x i (1000 - y):
x + (1000 - y) = (x - y) + y + (1000 - y) = (x - y) + 1000 = x - y
Widzimy że liczba (1000 - y) "pożycza" brakujące do nadmiaru y jedynek od
liczby x.

TODO: przykłady

Jeżeli zmrużymy teraz oczy, to być może uda nam się zauważyć pewną
analogię pomiędzy liczbą 999 a -1. Po pierwsze
1 + 999 = 0 vs 1 + (-1) = 0,
co więcej (-1) + (-1) = (-2), w wypadku 999 będzie to
999 + 999 = 998,
a liczba 998 ma to do siebie że 998 + 2 = 0, podobnie jak -2!
Jak się za chwilę przekonamy to nie przypadek, liczby 999 i -1 łączy
prosta zależność obie dają te same reszty modulo 1000.

NOTE: Umiemy obliczać reszty z dzielenia dla liczb dodatnich.
Żeby obliczyć resztę z dzielenia dla liczby ujemnej, musimy 
skorzystać z następującej zależności: Mówimy że liczby A i B mają
tą samą resztę z dzielenia modulo N jeżeli istnieje liczba 
całkowita k taka że:
A - B = kN
Na przykład 999 - (-1) = 1000 = 1*1000 a więc liczby (-1) i
999 mają tą samą resztę z dzielenia modulo N, podobnie liczby (-2) i 998.
O liczbach które mają taką samą resztę z dzielenia modulo N,
mówimy że przystają do siebie. Matematycy piszą w tym wypadku:
A == B (mod N)
co czytamy "A przystaje do B modulo N".
Przekładając to na prosty język, jeżeli chcemy obliczyć resztę z
dzielenia modulo N dla liczby ujemnej, to powinniśmy
kolejno dodawać do niej N aż otrzymamy liczbę dodatnią,
liczba ta będzie szukaną resztą.
Przykład:
-7 mod 3, -7 + 3 = -4, -4 + 3 = -1, -1 + 3 = 2
Szukana reszta to 2. Zauważmy (-7) - 2 = -9 = (-3) * 3.

Kontynuując naszą analogie, zobaczmy co się stanie jeżeli
utożsamimy liczbę 999 z -1, 998 z -2, i ogólnie 1000 - x z liczbą (-x).
Oczywiście chcielibyśmy mieć do dyspozycji również liczby dodatnie.
Żeby to osiągnąć liczby będziemy rezerwować zaczynając od zera:
    0
  1 0 999(-1)
2 1 0 999(-1) 998(-2)
\dots
499 \dots 2 1 0 999(-1) 998(-2) \dots 501(-499)
Okazuje się że kolejną liczbą graniczną będzie liczba 500, która ma tą
właściwość że 1000 - 500 = 500, jest więc ona pewnym odpowiednikiem zera
bo 1000 - 0 = 1000 = 0, lub ogólnie 1000 - x = x.
Na razie pominiemy 500 w naszych rozważaniach.

TODO: PIC licznik, gość zakłada okulary z napisem U10,
widzi liczby dodatnie i ujemne, zdejmuje okulary widzi liczby
dziesiętne.

Powyższy sposób reprezentacji liczb dodatnich i ujemnych będziemy
nazywać systemem uzupełnień do dziesięciu, w skrócie U10.
Dalej dla uniknięcia nieporozumień będziemy pisać
liczba 034_U10 (zwróć uwagę na index U10) jeżeli będziemy mówić o liczbie 34
zapisanej w kodowaniu U10. Przykładowo stan licznika 999 może reprezentować
liczbę 999 w systemie dziesiętnym lub liczbę -1_U10 w kodowaniu U10. 

Nasze pierwsze spostrzeżenie jest takie że bardzo prosto odróżnić
liczby dodatnie od ujemnych. Liczby dodanie mają w reprezentacji dziesiętnej
pierwszą cyfrę równą 0, 1, 2, 3 lub 4. 
Liczby ujemny rozpoczynają się cyframi 5, 6, 7, 8 i 9.
Równie prosto możemy zanegować liczbę, wystarczy odjąć jej wartość 
w reprezentacji dziesiętnej od tysiąca. Na przykład:
-(3_U10) = -(3) = 1000 - 3 = 997 = (-3)_U10
i analogicznie:
(-3)_U10 = -997 = 1000 - 997 = 3 = 3_U10

Dodawanie liczb dodaniach których suma jest mniejsza niż 500 nie
sprawia nam problemu, jeżeli suma przekracza 500 pojawia się nadmiar i wynik
okazuje się być ujemny. 
Dodanie dwóch liczb ujemnych również zachowuje się poprawnie
o ile suma nie będzie niższa niż -499 (a tak naprawdę -500):

(-x)_U10 + (-y)_U10 = (1000 - x) + (1000 - y) = 1000 + 1000 - (x + y)) =
= 1000 + (1000 - (x + y)) = 1000 - (x + y) = (-(x + y))_U10

bo jak pamiętamy (1000 + x) = x na naszym kalkulatorze.

Gdy suma dodawanych liczb ujemnych spada poniżej -500, pojawia się
nieznane nam jeszcze zjawisko niedomiaru (ang. underflow),
które skutkuje "przekręceniem się licznika w dół" i wynikowa sumą która
jest dodatnia.
TODO: Licznik przekręca się w dół -499 -> -500 -> 499 -> 498

Sprawdzenie że suma liczby dodatniej i ujemnej również jest poprawna,
o ile różnica mieści się z zakresie liczb reprezentowanych przez U10
pozostawiamy jako ćwiczenie dla czytelnika:
x_U10 + (-y)_U10 = x + (1000 - y) = ...

Mają zdefiniowaną negację i dodawanie w U10, możemy zdefiniować
odejmowanie jako:
x_U10 - y_U10 = x_U10 + (-y_U10)
Czyli gdy chcemy odjąć dwie liczby x i y, w reprezentacji U10
to najpierw negujemy y, a następnie dodajemy do tak otrzymanej wartości
liczbę x.
Przykład: z reprezentacją dziesiętną

Aby lepiej zrozumieć różnicę w reprezentacji dziesiętnej i U10,
wykonajmy kilka ćwiczeń zanim przejdziemy dalej.
TODO


% Plan
% - System dziesiętny 
% - System binarny
% - System trójkowy
% - Zbilansowany trójkowy
% - Bity i bajty, pamięć komputera, MSB, LSB, big/little endian
% - Operacje bitowe (i sztuczki jak zamiana zmiennych z XOR)
% - Arytmetyka za pomocą operacji bitowych, szybkie potęgowanie
